En el presente documento se detallan las condiciones para la acreditación de saberes propios tanto para materias técnico específicas como de formación científico tecnológica de la carrera de Técnico en Programación para estudiantes afectados al proyecto “Sistema de Gestión Institucional”.

Estas condiciones se especifican para cada uno de los estudiantes de manera particular a excepción de aquellos donde no se menciona de manera individualizada, en cuyo caso se entiende que aplica a todos y cada uno de ellos. 

Los estudiantes alcanzados por las especificaciones del presente documento, todos pertenecientes al 5to 3ra del ciclo lectivo 2024 de la carrera de programación, son:

    \begin{itemize}
        \item Julián Gonzalez.
        \item Bautista Izaguirre.
    \end{itemize}

\subsection{Materias no incluidas en el proyecto}

Estas materias no tienen relación directa con el proyecto “Sistema de Gestión Institucional” que los estudiantes están realizando y, por tanto, no se encuentren alcanzados por las condiciones del presente documento. Esto significa que deberán cursarse y acreditarse según los medios habituales incluyendo las condiciones de presentismo y promoción.

\subsubsection{Análisis Matemático (prof. Olmos)}
\subsubsection{Sistemas Digitales I (prof. Salimbeni)}

\subsection{Materias con acreditación parcial}

Son aquellas materias que si bien son alcanzadas por el proyecto, dadas sus condiciones particulares de promoción, serán acreditadas parcialmente mediante la entrega del "Sistema de Gestión Institucional" siendo necesario cumplimentar trabajos y/o entregas complementarios/as en la cursada.

A continuación se detallan las condiciones particulares acordadas con, y entre, los profesores para las materias:

\subsubsection{Base de Datos (prof. Balda)}       

El programa actualizado de la materia se encuentra en el Centro de Recursos Multimediales y se puede consultar ingresando la dirección en la barra del explorador o haciendo clic en el siguiente enlace:

\url{https://drive.google.com/drive/folders/1BbvXoqqAXC5Uq4ZXNdTD-EYfUl_R2YkI?usp=drive_link}.

La profesora hace entrega al equipo Directivo y Jerárquico un documento detallando su plan de trabajo para el desarrollo del proyecto "Sistema de Gestión Institucional". Este documento se adjunta al presente proyecto en el apartado de Anexos.

A modo de resumen se presentan los siguientes aspectos:
\paragraph{Expectativas de Logro}
Se espera que los estudiantes adquieran habilidades y conocimientos que les permitan: 
\begin{itemize}
    \item Crear una base de datos relacional.
    \item Implementar un motor de base de datos.
    \item Gestionar una base de datos.
    \item Comprender y aplicar comandos SQL. 
\end{itemize}


\paragraph{Evaluación y acreditación del docente}
Los estudiantes deberán presentar y exponer el proyecto "Sistema de Gestión Institucional" utilizando el vocabulario específico de la materia y respetando las siguientes condiciones:
\begin{itemize}
    \item Se deberá respetar el formato de la presentación indicado por el docente.
    \item Deberán incluir un informe de los temas indicados en el anexo.
    \item Deberán conocer la bibliografía indicada en el anexo y aplicar los fundamentos allí estudiados
\end{itemize}
\paragraph{Bibliografía} Los alumnos deberán utilizar el material indicado por el docente:
\begin{itemize}
    \item Marqués-Andrés, M. (2011). Bases de datos. Universitat Jaume I.
    \item  Oppel, A., \& Sheldon, R. (2008). SQL. McGraw-Hill Professional Publishing.
\end{itemize}

\subsubsection{Modelos y Sistemas (prof. Insaurralde)}.

El programa actualizado de la materia se encuentra en el Centro de Recursos Multimediales y se puede consultar ingresando la dirección en la barra del explorador o haciendo clic en el siguiente enlace:

\url{https://drive.google.com/drive/folders/1BbvXoqqAXC5Uq4ZXNdTD-EYfUl_R2YkI?usp=drive_link}.

Este espacio se relaciona íntegramente con el proyecto y tanto la dinámica como los objetivos propuestos por el docente se alinean perfectamente con los requerimientos del sistema que los alumnos deben desarrollar. Por eso la acreditación del espacio estará completamente a cargo del docente.

\paragraph{Expectativas de Logro} Finalizada la cursada del espacio los alumnos tendrán conocimientos y capacidades para:

\begin{itemize}
    \item Realizar programación estructurada y secuencial. 
    \item Utilizar herramientas específicas para la diagramación de flujo de datos.
    \item Comprender y realizar ciclos anidados mediante modelos ER y formas normales utilizando PK y FK.
    \item Realizar la documentación necesaria para un sistema informático.
\end{itemize}

\paragraph{Evaluación} Para el visado final, los alumnos deberán entregar al docente:
\begin{itemize}
    \item Manual del Sistema o del Programador.
    \item Manual de Usuario.
\end{itemize}

\paragraph{Bibliografía} Los alumnos tendrán acceso al siguiente material bibliográfico:

\begin{itemize}
    \item Santiago Ramírez (1999). Teoría General de los sistemas de Bertalanffy. Universidad Autónoma de México. Capítulo 3 y 4.
    \item Bonatti(2011). Teoría de la decisión. Editorial Prearson.
    \item Pablo Sznajdleder(2012). Algoritmos a fondo con implementaciones en c y java. Editorial Alfaomega.
\end{itemize}


\subsection{Materias con acreditación conjunta}

Son aquellas materias que se encuentran alcanzadas en su totalidad por el proyecto. Estas materias se acreditan mediante la presentación del "Sistema de Gestión Institucional" cumpliendo con la totalidad de los objetivos propuestos.

\paragraph{Evaluación y acreditación} El equipo Directivo y los docentes cuyos espacios se incluyen en los apartados 3.2 y 3.3 del presente documento se reúnen en comisión al finalizar el segundo cuatrimestre para analizar la funcionalidad del sistema. Durante el período de intensificación de diciembre se realizarán pruebas de funcionamiento llevando adelante la depuración del sistema hasta no presentar errores críticos o de gravedad. Al finalizar el presente ciclo lectivo los módulos solicitados en el apartado 4 deberán ser completamente funcionales.



\subsubsection{Laboratorio de Diseño de Bases de Datos (prof. Sanfelice - prof. Ganduglia)} 

El programa actualizado de la materia se encuentra en el Centro de Recursos Multimediales y se puede consultar ingresando la dirección en la barra del explorador o haciendo clic en el siguiente enlace:

\url{https://drive.google.com/drive/folders/1BbvXoqqAXC5Uq4ZXNdTD-EYfUl_R2YkI?usp=drive_link}.

Los profesores entregan al equipo Directivo y Jerárquico un plan de trabajo basado en proyectos. Este plan divide en grupos al total de los alumnos que cursan la materia y a cada grupo se le asigna una problemática diferente a resolver. El proyecto "Sistema de Gestión Institucional" es totalmente compatible con la modalidad de trabajo planteada por los docentes.

A modo resumen se presentan los siguientes aspectos:
\paragraph{Expectativas de Logro}
Se espera que los estudiantes adquieran habilidades y conocimientos que les permitan: 

\begin{itemize}
    \item Crear de bases de datos.
    \item Comprender y aplicar el modelo relacional.
    \item Mantener y gestionar bases de datos propias o de terceros.
    \item Crear módulos e interfaces para el acceso seguro de usuarios.
    \item Interpretar y crear consultas mediante lenguaje SQL.
    \item Implementar y mantener servidores y motores de bases de datos. 
\end{itemize}

\paragraph{Evaluación y acreditación}

Los estudiantes deberán presentar y exponer el proyecto "Sistema de Gestión Institucional" depurado y sin errores críticos, o que revistan gravedad, en su funcionamiento. Los docentes de los espacios incluidos en los apartados 3.2 y 3.3, especialistas del área, analizan en conjunto con el equipo Directivo y Jerarquico los aspectos técnicos y la correcta implementación del sistema así como también la utilización de las buenas prácticas de programación.

\paragraph{Bibliografía} Los docentes recomiendan a los estudiantes el siguiente material de consulta:

\begin{itemize}
    \item Aguirre Juan, (2009). Bases de datos en MySQL 
    \item Sánchez, J. (2004). Principios sobre bases de datos relacionales. Informe, Creative Commons, 11, 20.
    \item Minera, F. (2011). Desarrollo PHP y MySQL. USERSHOP.
\end{itemize}
   %Laboratorio de base de datos.
\subsubsection{Laboratorio de Programación (prof. Sanfelice - prof. Ganduglia)}

El programa actualizado de la materia se encuentra en el centro de recursos multimedia y puede ser accedido ingresando la dirección en la barra del explorador o haciendo click en el link :


\url{https://drive.google.com/drive/folders/1BbvXoqqAXC5Uq4ZXNdTD-EYfUl_R2YkI?usp=drive_link}



Los profesores entregan al equipo Directivo y Jerárquico un plan de trabajo basado en proyectos, donde el total de los alumnos que cursan la materia se dividen en grupos, a los mismos se les asignan diferentes proyectos y situaciones problemáticas a resolver, por lo cual el "Sistema de Gestión Institucional", es totalmente compatible con la modalidad de trabajo planteada por los docentes.

A modo resumen se presentan los siguientes aspectos:
\paragraph{Espectativas de Logro}
Se espera que los estudiantes adquieran habilidades y conocimientos que les permitan: 

\begin{itemize}
    \item Utilizar lenguajes de programación estructurados.
    \item Utilizar lenguajes de programación orientados a objetos.
    \item Resolver problemas sencillos mediante la creación de programas / algoritmos.
    \item Mantener, Revisar y depurar software.
    \item Utilizar herramientas para el control de versiones.
    \item Desarrollar soluciones de gestión a nivel macro aplicando todas las herramientas de desarrollo actuales. 
\end{itemize}

\paragraph{Evaluación y acreditación}

Los estudiantes deberán presentar y exponer el Proyecto Integrador Final "Sistema de Gestión Institucional", depurado y sin errores graves o críticos en el funcionamiento del mismo. Los docentes especialista del área analizan en conjunto con el equipo directivo y Jerarquico los aspectos técnicos y la correcta implementación, así como el uso de buenas prácticas de programación.

\paragraph{Bibliografía}
\begin{itemize}
    \item  (Luis Joyanes Aguilar, Ignacio Zahonero Martínez). Programación en C, C++, Java y UML - McGRAW-HILL Education

\end{itemize}     %Laboratorio de Programación.
\subsubsection{Laboratorio de Diseño Web (prof. González - prof. Ganduglia)}
El programa actualizado de la materia se encuentra en el Centro de Recursos Multimediales y se puede consultar ingresando la dirección en la barra del explorador o haciendo clic en el siguiente enlace:

\url{https://drive.google.com/drive/folders/1BbvXoqqAXC5Uq4ZXNdTD-EYfUl_R2YkI?usp=drive_link}.

Los profesores entregan al equipo Directivo y Jerárquico un plan de trabajo basado en proyectos, donde el total de los alumnos que cursan la materia se dividen en grupos, a los mismo se les asignan diferentes proyectos a resolver, por lo cual el "Sistema de Gestión Institucional", es totalmente compatible con la modalidad de trabajo planteada por los docentes.

A modo resumen se presentan los siguientes aspectos:
\paragraph{Expectativas de Logro}
Se espera que los estudiantes adquieran habilidades y conocimientos que les permitan: 

\begin{itemize}
    \item Creación de páginas web mediante el uso del HTML.
    \item Creación de estilos mediante el uso de CSS.
    \item Publicación de páginas y uso de servidores APACHE.
    \item Creación de formularios e integración con bases de datos.
    \item Realización de consultas mediante el uso de lenguaje PHP.

\end{itemize}

\paragraph{Evaluación y acreditación}

Los estudiantes deberán presentar y exponer el Proyecto Integrador Final "Sistema de Gestión Institucional", depurado y sin errores graves o críticos en el funcionamiento del mismo. Los docentes especialista del área analizan en conjunto con el equipo directivo y Jerarquico los aspectos técnicos y la correcta implementación así como el uso de buenas prácticas de programación.

\paragraph{Bibliografía} Los docentes recomiendan a los estudiantes el siguiente material de consulta:

\begin{itemize}
    \item Beati, H. (2020). HTML5 y CSS3 para diseñadores. Marcombo.
\end{itemize}
    %Laboratorio de Diseño Web.
\subsubsection{Laboratorio de Redes Informáticas (prof. Sanfelice - prof. Figueroa)}.

El programa actualizado de la materia se encuentra en el Centro de Recursos Multimediales y se puede consultar ingresando la dirección en la barra del explorador o haciendo clic en el siguiente enlace:

\url{https://drive.google.com/drive/folders/1BbvXoqqAXC5Uq4ZXNdTD-EYfUl_R2YkI?usp=drive_link}

Los profesores entregan al equipo Directivo y Jerárquico un plan de trabajo basado en proyectos. Este plan divide en grupos al total de los alumnos que cursan la materia y a cada grupo se le asigna una problemática diferente a resolver. El proyecto "Sistema de Gestión Institucional" es totalmente compatible con la modalidad de trabajo planteada por los docentes.

A modo de resumen se presentan los siguientes aspectos:
\paragraph{Expectativas de Logro}
Se espera que los estudiantes adquieran habilidades y conocimientos que les permitan: 

\begin{itemize}
    \item item
\end{itemize}

\paragraph{Evaluación y acreditación}

Los estudiantes deberán presentar y exponer el proyecto "Sistema de Gestión Institucional" depurado y sin errores críticos, o que revistan gravedad, en su funcionamiento. Los docentes de los espacios incluidos en los apartados 3.2 y 3.3, especialistas del área, analizan en conjunto con el equipo Directivo y Jerarquico los aspectos técnicos y la correcta implementación del sistema así como también la utilización de las buenas prácticas de programación.

\paragraph{Bibliografía} Los docentes recomiendan a los estudiantes el siguiente material de consulta:
\begin{itemize}
    \item  

\end{itemize}
    %Laboratorio de redes informáticas

