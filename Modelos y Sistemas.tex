\subsubsection{Modelos y Sistemas (prof. Insaurralde)}.

El programa actualizado de la materia se encuentra en el Centro de Recursos Multimediales y se puede consultar ingresando la dirección en la barra del explorador o haciendo clic en el siguiente enlace:

\url{https://drive.google.com/drive/folders/1BbvXoqqAXC5Uq4ZXNdTD-EYfUl_R2YkI?usp=drive_link}.

Este espacio se relaciona íntegramente con el proyecto y tanto la dinámica como los objetivos propuestos por el docente se alinean perfectamente con los requerimientos del sistema que los alumnos deben desarrollar. Por eso la acreditación del espacio estará completamente a cargo del docente.

\paragraph{Expectativas de Logro} Finalizada la cursada del espacio los alumnos tendrán conocimientos y capacidades para:

\begin{itemize}
    \item Realizar programación estructurada y secuencial. 
    \item Utilizar herramientas específicas para la diagramación de flujo de datos.
    \item Comprender y realizar ciclos anidados mediante modelos ER y formas normales utilizando PK y FK.
    \item Realizar la documentación necesaria para un sistema informático.
\end{itemize}

\paragraph{Evaluación} Para el visado final, los alumnos deberán entregar al docente:
\begin{itemize}
    \item Manual del Sistema o del Programador.
    \item Manual de Usuario.
\end{itemize}

\paragraph{Bibliografía} Los alumnos tendrán acceso al siguiente material bibliográfico:

\begin{itemize}
    \item Santiago Ramírez (1999). Teoría General de los sistemas de Bertalanffy. Universidad Autónoma de México. Capítulo 3 y 4.
    \item Bonatti(2011). Teoría de la decisión. Editorial Prearson.
    \item Pablo Sznajdleder(2012). Algoritmos a fondo con implementaciones en c y java. Editorial Alfaomega.
\end{itemize}
