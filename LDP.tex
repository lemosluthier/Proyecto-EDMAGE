\subsubsection{Laboratorio de Programación (prof. Sanfelice - prof. Ganduglia)}

El programa actualizado de la materia se encuentra en el centro de recursos multimedia y puede ser accedido ingresando la dirección en la barra del explorador o haciendo click en el link :


\url{https://drive.google.com/drive/folders/1BbvXoqqAXC5Uq4ZXNdTD-EYfUl_R2YkI?usp=drive_link}



Los profesores entregan al equipo Directivo y Jerárquico un plan de trabajo basado en proyectos, donde el total de los alumnos que cursan la materia se dividen en grupos, a los mismos se les asignan diferentes proyectos y situaciones problemáticas a resolver, por lo cual el "Sistema de Gestión Institucional", es totalmente compatible con la modalidad de trabajo planteada por los docentes.

A modo resumen se presentan los siguientes aspectos:
\paragraph{Espectativas de Logro}
Se espera que los estudiantes adquieran habilidades y conocimientos que les permitan: 

\begin{itemize}
    \item Utilizar lenguajes de programación estructurados.
    \item Utilizar lenguajes de programación orientados a objetos.
    \item Resolver problemas sencillos mediante la creación de programas / algoritmos.
    \item Mantener, Revisar y depurar software.
    \item Utilizar herramientas para el control de versiones.
    \item Desarrollar soluciones de gestión a nivel macro aplicando todas las herramientas de desarrollo actuales. 
\end{itemize}

\paragraph{Evaluación y acreditación}

Los estudiantes deberán presentar y exponer el Proyecto Integrador Final "Sistema de Gestión Institucional", depurado y sin errores graves o críticos en el funcionamiento del mismo. Los docentes especialista del área analizan en conjunto con el equipo directivo y Jerarquico los aspectos técnicos y la correcta implementación, así como el uso de buenas prácticas de programación.

\paragraph{Bibliografía}
\begin{itemize}
    \item  (Luis Joyanes Aguilar, Ignacio Zahonero Martínez). Programación en C, C++, Java y UML - McGRAW-HILL Education

\end{itemize}