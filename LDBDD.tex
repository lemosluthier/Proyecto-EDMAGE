
\subsubsection{Laboratorio de Diseño de Bases de Datos (prof. Sanfelice - prof. Ganduglia)} 

El programa actualizado de la materia se encuentra en el centro de recursos multimedia y puede ser accedido ingresando la dirección en la barra del explorador o haciendo click en el link : 

\url{https://drive.google.com/drive/folders/1BbvXoqqAXC5Uq4ZXNdTD-EYfUl_R2YkI?usp=drive_link}.


Los profesores entregan al equipo Directivo y Jerárquico un plan de trabajo basado en proyectos, donde el total de los alumnos que cursan la materia se dividen en grupos, a los mismo se les asignan diferentes proyectos a resolver, por lo cual el "Sistema de Gestión Institucional de la EEST1", es totalmente compatible con la modalidad de trabajo planteada por los docentes.

A modo resumen se presentan los siguientes aspectos:
\paragraph{Expectativas de Logro}
Se espera que los estudiantes adquieran habilidades y conocimientos que les permitan: 

\begin{itemize}
    \item Crear de bases de datos.
    \item Comprender y Aplicar el modelo relacional.
    \item Mantener y gestionar bases de datos propias o de terceros.
    \item Crear módulos e interfaces para el acceso seguro de usuarios.
    \item Interpretar y crear consultas mediante lenguaje SQL.
    \item Implementar y mantener servidores y motores de bases de datos. 
\end{itemize}

\paragraph{Evaluación y acreditación}

Los estudiantes deberán presentar y exponer el Proyecto Integrador Final "Sistema de Gestión Institucional", depurado y sin errores graves o críticos en el funcionamiento del mismo. Los docentes especialista del área analizan en conjunto con el equipo directivo y Jerarquico los aspectos técnicos y la correcta implementación así como el uso de buenas prácticas de programación.

\paragraph{Bibliografía} Los docentes recomiendan a los estudiantes el siguiente material de consulta:

\begin{itemize}
    \item Aguirre Juan, (2009). Bases de datos en MySQL 
    \item Sánchez, J. (2004). Principios sobre bases de datos relacionales. Informe, Creative Commons, 11, 20.
    \item Minera, F. (2011). Desarrollo PHP y MySQL. USERSHOP.
\end{itemize}