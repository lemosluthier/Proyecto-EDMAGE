Visto el ACTA-2024-28451026, con fecha de agosto de 2024, donde se indican los procedimientos de intervención con  alumnos que requieren estrategias didácticas y pedagógicas particularizadas, en especial para aquellos con trayectorias avanzadas.

Considerando lo determinado en la Res. CFE 14/07. que dice: “La educación técnico profesional introduce a los estudiantes, jóvenes y adultos, en un recorrido de profesionalización a partir del acceso a una base de conocimientos y de habilidades profesionales que les permita su inserción en áreas ocupacionales cuya complejidad exige haber adquirido una formación general, una cultura científico - tecnológica de base a la par de una formación técnica específica de carácter profesional, así como continuar aprendiendo durante toda su vida. Procura, además, responder a las demandas y necesidades del contexto socio productivo en el cual se desarrolla, con una mirada integral y prospectiva que excede a la preparación para el desempeño de puestos de trabajo u oficios específicos”.

Atendiendo a la Res. CFE Nro. 148/11 Anexo I Marco de referencia para procesos de homologación de títulos del nivel secundario Sector Informático, donde se establecen las bases del perfil profesional y que explicitan las competencias del técnico en programación.

Y observando que un grupo de alumnos que se encuentran cursando el 5to año de la carrera Técnico en Programación cuentan con inquietudes particulares y manifiestan gran capacidad de aprendizaje, autonomía y responsabilidad más allá de lo esperado para su rango etario y que ya han participado en proyectos ganadores de feria de ciencias valorados tanto dentro como fuera de la comunidad de la Técnica 1.


