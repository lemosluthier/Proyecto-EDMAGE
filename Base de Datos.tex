\subsubsection{Base de Datos (prof. Balda)}       

El programa actualizado de la materia se encuentra en el Centro de Recursos Multimediales y se puede consultar ingresando la dirección en la barra del explorador o haciendo clic en el siguiente enlace:

\url{https://drive.google.com/drive/folders/1BbvXoqqAXC5Uq4ZXNdTD-EYfUl_R2YkI?usp=drive_link}.

La profesora hace entrega al equipo Directivo y Jerárquico un documento detallando su plan de trabajo para el desarrollo del proyecto "Sistema de Gestión Institucional". Este documento se adjunta al presente proyecto en el apartado de Anexos.

A modo de resumen se presentan los siguientes aspectos:
\paragraph{Expectativas de Logro}
Se espera que los estudiantes adquieran habilidades y conocimientos que les permitan: 
\begin{itemize}
    \item Crear una base de datos relacional.
    \item Implementar un motor de base de datos.
    \item Gestionar una base de datos.
    \item Comprender y aplicar comandos SQL. 
\end{itemize}


\paragraph{Evaluación y acreditación del docente}
Los estudiantes deberán presentar y exponer el proyecto "Sistema de Gestión Institucional" utilizando el vocabulario específico de la materia y respetando las siguientes condiciones:
\begin{itemize}
    \item Se deberá respetar el formato de la presentación indicado por el docente.
    \item Deberán incluir un informe de los temas indicados en el anexo.
    \item Deberán conocer la bibliografía indicada en el anexo y aplicar los fundamentos allí estudiados
\end{itemize}
\paragraph{Bibliografía} Los alumnos deberán utilizar el material indicado por el docente:
\begin{itemize}
    \item Marqués-Andrés, M. (2011). Bases de datos. Universitat Jaume I.
    \item  Oppel, A., \& Sheldon, R. (2008). SQL. McGraw-Hill Professional Publishing.
\end{itemize}
