Los estudiantes involucrados en el presente proyecto estarán encargados de desarrollar un "Sistema de Gestión Institucional". Se prevé un plazo de tres años para su finalización. La extensión del proyecto incluye el 5to, 6to y 7mo año de los estudiantes interesados.

En esta primera etapa de desarrollo deberán entregar un sistema parcial formado por los siguientes módulos:

\begin{itemize}
    \item Módulo Pañol.
    \item Módulo de Inscripción.
    \item Módulo de Asistencia.
    \item Módulo ABM de alumnos y profesores
\end{itemize}

Cada módulo tendrá una fecha de entraga previamente establecida. El plazo de entrega final para este sistema parcial será hasta la finalización del ciclo lectivo 2024.

Los módulos deben ser accesibles a través de un portal WEB alojado en el servidor de la institución. Solo podrán acceder usuarios que se encuentren dentro de la intranet.

\subsection{Módulo Pañol}

\subsection{Módulo de Inscripción}

Los aspirantes a vacantes para el primer año realizan una preinscripción durante el mes de diciembre. Este módulo debe permitir a las familias realizar esta preinscripción a través de terminales instaladas en la escuela. La inscripción efectiva se completará con la incorporación de la documentación respaldatoria que conformará el lejago digital.

Este módulo debe encontrarse operativo antes de este período.

\subsection{Módulo de Asistencia}

Este módulo permite tomar asistencia a los alumnos que se encuentren inscriptos en las distintas materias dentro de sus correspondientes años. La implementación de este módulo está prevista para el ciclo lectivo 2025.

Este módulo debe encontrarse operativo antes de dar inicio al ciclo lectivo 2025.

Este registro debe contemplar la generación de informes que den cuenta de las distintas ponderaciones de asistencia.

\subsection{Módulo ABM}

Este módulo debe de contener todos los cámpos necesarios para poder suplantar o respaldar toda la información que se encuentra contenida en los legajos de los alumnos y los docentes.

En una primera instancia el grupo de alumnos deberá coordinar con el equipo directivo el ingreso de la totalidad de los alumnos y verificar que los mismos cuentan con perfiles de usuario que no les permitan editar o ingresar ningún dato.
