Los alumnos involucrados en el presente proyecto estarán encargados de presentar un sistema de
gestión institucional, se preveen para la culminación del mismo un período de tres años 
incluyendo 5to, 6to y 7mo.

Para esta primera etapa se debe de entregar finalizado el ciclo lectivo un sistema parcial que contenga los siguientes módulos:

\begin{itemize}
    \item Módulo Pañol.
    \item Módulo de Inscripción.
    \item Módulo de Asistencia.
    \item Módulo ABM de alumnos y profesores
\end{itemize}

Los módulos deben de ser accesibles a través de un portal WEB alojado en el servidor de la institución y solo por usuarios que se encuentren dentro de la intranet.

\subsection{Módulo ABM}

Este módulo debe de contener todos los cámpos necesarios para poder suplantar o respaldar toda la información que se encuentra contenida en los legajos de los alumnos y los docentes.

En una primera instancia el grupo de alumnos deberá coordinar con el equipo directivo el ingreso de la totalidad de los alumnos y verificar que los mismos cuentan con perfiles de usuario que no les permitan editar o ingresar ningún dato.

\subsection{Módulo de Inscripción}

En el período de diciembre los aspirantes a vacantes para el primer año se presentan en la escuela. La intención con este módulo es permiter una preinscripción por parte de las familias a través de terminales instaladas en la escuela para luego se incorpore la documentación respaldatoria que conformará el lejago digital.

\subsection{Módulo de Asistencia}

Antes de comenzar el ciclo lectivo 2025, el sistema deberá contar con la posibilidad de tomar asistencia a los alumnos que se encuentran inscriptos en las distintas materias en suss correspondientes años.

Este registro debe de poder contemplar la generación de informes en los cuales se deberá discriminar las distintas ponderaciones de asistencia.