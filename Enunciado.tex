Los estudiantes involucrados en el presente proyecto estarán encargados de desarrollar un "Sistema de Gestión Institucional". Se prevé un plazo de tres años para su finalización. La extensión del proyecto incluye el 5to, 6to y 7mo año de los estudiantes interesados.

En esta primera etapa de desarrollo deberán entregar un sistema parcial formado por los siguientes módulos:

\begin{itemize}
    \item Módulo Pañol.
    \item Módulo de Inscripción.
    \item Módulo de Asistencia.
    \item Módulo ABM de alumnos y profesores
\end{itemize}

Cada módulo tendrá una fecha de entraga previamente establecida. El plazo de entrega final para este sistema parcial será hasta la finalización del ciclo lectivo 2024.

Los módulos deben ser accesibles a través de un portal WEB alojado en el servidor de la institución. Solo podrán acceder usuarios que se encuentren dentro de la intranet.

\subsection{Módulo Pañol}

Diariamente en la escuela, los alumnos que cursan materias con modalidad de taller solicitan herramientas e instrumental para avanzar con los proyectos y actividades propuestas. Si bien las herramientas se encuentran a cuidado de los EMATP pañol, se hace evidente la necesidad de herramientas de gestión de inventario a la hora de mejorar las tareas asociadas al prestamo de los elementos siendo los puntos a mejorar:

\begin{itemize}
    \item Búsqueda rápida de herramientas. Dado que la ubicación y cantidad disponibles están al alcance de todos los usuarios del sistema.
    \item Control de inventario.
    \item Mejora de los tiempos de altas y bajas de herramientas e instrumental.
    \item Mejora de los tiempos de solicitud de herramientas e instrumental.
    \item Mejora de los tiempos de entrega y devolución de herramientas e instrumental
\end{itemize}

Para lograr los objetivos propuestos :

\subsubsection{Módulo Pañol (solicitud)}
Se solicita a los estudiantes un módulo que permita a traves de un dispositivo movil, o una computadora, realizar búsquedas de herramientas y solicitar y/o reservar los elementos del inventario.
\subsubsection{Módulo Pañol (entrega)}
Se solicita a los estudiantes un módulo que permita al EMATP pañol a partir de una computadora instalada en el pañol o un dispositivo movil, recibir solicitudes y realizar las entregas de los elementos requeridos.
\subsubsection{Módulo Pañol (devolución)}
Se solicita a los estudiantes un módulo que permita a los docentes y al EMATP pañol a partir de una computadora instalada en dicho pañol o un dispositivo movil, registrar las devoluciones de los elementos prestados.

\subsubsection{Módulo Pañol (invetario)}
Se solicita a los estudiantes un módulo que permita al EMATP pañol realizar controles periódicos del estado del inventario pudiendo asentar informes de la situación.


\subsection{Módulo de Inscripción}

Los aspirantes a vacantes para el primer año realizan una preinscripción durante el mes de diciembre. Este módulo debe permitir a las familias realizar esta preinscripción a través de terminales instaladas en la escuela. La inscripción efectiva se completará con la incorporación de la documentación respaldatoria que conformará el lejago digital.

Este módulo debe encontrarse operativo antes de este período.

\subsection{Módulo de Asistencia}

Este módulo permite tomar asistencia a los alumnos que se encuentren inscriptos en las distintas materias dentro de sus correspondientes años. La implementación de este módulo está prevista para el ciclo lectivo 2025.

Este módulo debe encontrarse operativo antes de dar inicio al ciclo lectivo 2025.

Este registro debe contemplar la generación de informes que den cuenta de las distintas ponderaciones de asistencia.

\subsection{Módulo ABM}

Este módulo deberá contener todos los cámpos necesarios para poder agregar, modificar, corregir y respaldar toda la información que se encuentra contenida en los legajos de los alumnos y los docentes.

En una primera instancia, el grupo de desarrollo deberá coordinar con el equipo Directivo y Jerárquico la carga de datos de todos los alumnos y verificar que todos cuentan con perfiles de usuario. Estos perfiles no deben tener permisos para ingresar ni editar ningún dato.
