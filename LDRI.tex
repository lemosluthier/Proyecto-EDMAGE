\subsubsection{Laboratorio de Redes Informáticas (prof. Sanfelice - prof. Figueroa)}.

El programa actualizado de la materia se encuentra en el Centro de Recursos Multimediales y se puede consultar ingresando la dirección en la barra del explorador o haciendo clic en el siguiente enlace:

\url{https://drive.google.com/drive/folders/1BbvXoqqAXC5Uq4ZXNdTD-EYfUl_R2YkI?usp=drive_link}

Los profesores entregan al equipo Directivo y Jerárquico un plan de trabajo basado en proyectos. Este plan divide en grupos al total de los alumnos que cursan la materia y a cada grupo se le asigna una problemática diferente a resolver. El proyecto "Sistema de Gestión Institucional" es totalmente compatible con la modalidad de trabajo planteada por los docentes.

A modo de resumen se presentan los siguientes aspectos:
\paragraph{Expectativas de Logro}
Se espera que los estudiantes adquieran habilidades y conocimientos que les permitan: 

\begin{itemize}
    \item Entender las capas del modelo OSI y TCP/IP para diagnosticar y reparar problemas de red.
    \item Identificar las partes de una dirección IP y en qué consiste la numeración IPv4 y IPv6.
    \item Utilizar de forma correcta las herramientas para construir un cable de red UTP así cómo el código de colores para armar una ficha RJ-45.
    \item Conocer para que se utilizan los siguientes dispositivos de red: hub, switch y router.
    \item Implementar el protocolo STP (Protocolo de Árbol de Expansión) para evitar el loop entre dispositivos Hub o Switch.
    \item Utilizar el software de control de versiones GIT en un entorno de desarrollo
    Identificar los diferentes protocolos cuando realiza un auditoría de red (Sniffing).
    \item Utilizar comandos para administrar un servidor Web de forma remota
    Migrar un sitio web a un servidor: físico o virtual.
    \item Utilizar los protocolos de los siguientes servicios de red: FTP, DNS, HTTP y SMTP.
    \item Detallar los requisitos de hardware y software de un servidor web de producción.  
    
\end{itemize}

\paragraph{Evaluación y acreditación}

Los estudiantes deberán presentar y exponer el proyecto "Sistema de Gestión Institucional" depurado y sin errores críticos, o que revistan gravedad, en su funcionamiento. Los docentes de los espacios incluidos en los apartados 3.2 y 3.3, especialistas del área, analizan en conjunto con el equipo Directivo y Jerarquico los aspectos técnicos y la correcta implementación del sistema así como también la utilización de las buenas prácticas de programación.

\paragraph{Bibliografía} Los docentes recomiendan a los estudiantes el siguiente material de consulta:
\begin{itemize}
    \item Redes, guía de referencia (Abad Sergio, 2006), 
    \item Fundamentos básicos de redes de área local (López Evaristo, 2007), 
    \item Seguridad en redes informáticas (Cortes Juán, 2005)


\end{itemize}
