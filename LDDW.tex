\subsubsection{Laboratorio de Diseño Web (prof. González - prof. Ganduglia)}
El programa actualizado de la materia se encuentra en el centro de recursos multimedia y puede ser accedido ingresando la dirección en la barra del explorador o haciendo click en el link :


\url{https://drive.google.com/drive/folders/1BbvXoqqAXC5Uq4ZXNdTD-EYfUl_R2YkI?usp=drive_link}.



Los profesores entregan al equipo Directivo y Jerárquico un plan de trabajo basado en proyectos, donde el total de los alumnos que cursan la materia se dividen en grupos, a los mismo se les asignan diferentes proyectos a resolver, por lo cual el "Sistema de Gestión Institucional", es totalmente compatible con la modalidad de trabajo planteada por los docentes.

A modo resumen se presentan los siguientes aspectos:
\paragraph{Expectativas de Logro}
Se espera que los estudiantes adquieran habilidades y conocimientos que les permitan: 

\begin{itemize}
    \item Creación de páginas web mediante el uso del HTML.
    \item Creación de estilos mediante el uso de CSS.
    \item Publicación de páginas y uso de servidores APACHE.
    \item Creación de formularios e integración con bases de datos.
    \item Realización de consultas mediante el uso de lenguaje PHP.

\end{itemize}

\paragraph{Evaluación y acreditación}

Los estudiantes deberán presentar y exponer el Proyecto Integrador Final "Sistema de Gestión Institucional", depurado y sin errores graves o críticos en el funcionamiento del mismo. Los docentes especialista del área analizan en conjunto con el equipo directivo y Jerarquico los aspectos técnicos y la correcta implementación así como el uso de buenas prácticas de programación.

\paragraph{Bibliografía} Los docentes recomiendan a los estudiantes el siguiente material de consulta:

\begin{itemize}
    \item Beati, H. (2020). HTML5 y CSS3 para diseñadores. Marcombo.
\end{itemize}