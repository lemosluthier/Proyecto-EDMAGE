\subsubsection{Laboratorio de Diseño Web (prof. González - prof. Ganduglia)}
El programa actualizado de la materia se encuentra en el Centro de Recursos Multimediales y se puede consultar ingresando la dirección en la barra del explorador o haciendo clic en el siguiente enlace:

\url{https://drive.google.com/drive/folders/1BbvXoqqAXC5Uq4ZXNdTD-EYfUl_R2YkI?usp=drive_link}.

Los profesores entregan al equipo Directivo y Jerárquico un plan de trabajo basado en proyectos. Este plan divide en grupos al total de los alumnos que cursan la materia y a cada grupo se le asigna una problemática diferente a resolver. El proyecto "Sistema de Gestión Institucional" es totalmente compatible con la modalidad de trabajo planteada por los docentes.

A modo de resumen se presentan los siguientes aspectos:
\paragraph{Expectativas de Logro}
Se espera que los estudiantes adquieran habilidades y conocimientos que les permitan: 

\begin{itemize}
    \item Crear páginas web mediante el uso del HTML.
    \item Crear estilos mediante el uso de CSS.
    \item Publicar páginas y administrar servidores APACHE.
    \item Crear formularios e integrarlos con bases de datos.
    \item Realizar consultas mediante el uso de lenguaje PHP.

\end{itemize}

\paragraph{Evaluación y acreditación}

Los estudiantes deberán presentar y exponer el proyecto "Sistema de Gestión Institucional" depurado y sin errores críticos, o que revistan gravedad, en su funcionamiento. Los docentes de los espacios incluidos en los apartados 3.2 y 3.3, especialistas del área, analizan en conjunto con el equipo Directivo y Jerarquico los aspectos técnicos y la correcta implementación del sistema así como también la utilización de las buenas prácticas de programación.

\paragraph{Bibliografía} Los docentes recomiendan a los estudiantes el siguiente material de consulta:

\begin{itemize}
    \item Beati, H. (2020). HTML5 y CSS3 para diseñadores. Marcombo.
\end{itemize}
