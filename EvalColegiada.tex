Al finalizar la cursada regular, los estudiantes deberán hacer una presentación del trabajo realizado donde expondrán el funcionamiento del "Sistema de Gestión Institucional" frente al equipo Directivo, el Jefe de Área y los docentes de los espacios consignados en los apartados 3.2 y 3.3. Esta presentación deberá incluir:

\begin{itemize}
    \item El Manual de Usuario impreso.
    \item El Manual del Programador impreso.
    \item Una presentación multimedia.
    \item Una lista de tutoriales. 
\end{itemize}

\subsection{Manual de Usuario}

El Manual de Usuario será un requisito para la acreditación del espacio \textbf{Modelos y Sistemas}. Este deberá contener de manera  sistemática y ordenada la guía para el uso de todas las funciones de cada módulo del sistema organizadas por perfil.

\subsection{Manual del Programador}

El Manual del Programador será un requisito para la acreditación del espacio \textbf{Modelos y Sistemas}. Este deberá contener de manera sistemática y ordenada la documentación requerida para el mantenimiento de cada módulo del sistema así como también toda la información necesaria para la incorporación de nuevos módulos o funciones adicionales.

\subsection{Presentación multimedia}

La presentación debe proporcionar una explicación pormenorizada del funcionamiento de todo el sistema en la cual se ponga de mnifiesto el conocimiento técnico formal aplcado así como también el uso de vocabulario específico. 

\subsection{Lista de tutoriales}

La lista de tutoriales consistirá de una colección de videos organizados por tipo de operación y perfil. Estos serán utilizados para la capacitación del equipo Jerárquico y del cuerpo de preceptores ya que serán los primeros usuarios que tendrá el sistema. 
